% Robin Prillwitz 2022

\usepackage[
  acronym, toc
]{glossaries}
\usepackage[automake, nonumberlist, nogroupskip]{glossaries-extra}

\renewcommand*{\glsclearpage}{} % remove gloassary pagebreak
\setabbreviationstyle[acronym]{long-short}
\setglossarystyle{long}
\renewenvironment{theglossary}%
  {\begin{longtable}[l]{p{.5\glsdescwidth}p{\glsdescwidth}}}%
  {\end{longtable}}
\renewcommand{\glsnamefont}[1]{\textbf{#1}}
% \renewcommand*{\glstextformat}[1]{\textit{#1}} % cursive acronyms

% enter your own acronyms and glossary entries here
% use in the text using the \gls{} and \acrshort{} or \acr... family of commands

% --------------------------------- Acronyms -------------------------------- %
\newacronym{acr}{ACR}{Acronym}
\newacronym{acr1}{LangesAkronymWarumAuchImmerNurMalSoZumTestWeilIchSoLustigBin}{Ein sehr langes Akronym}
\newacronym{acr2}{ACR 2}{Eine sehr lange Beschreibung eines Akronyms. Das ist dazu da um zu schauen, wie sich das System verhält und wie un-ansichtlich das Akronymverzeichnis in der sogenannten Frontmatter ist.}

% --------------------------------- Glossary -------------------------------- %
\newglossaryentry{glossary}{name={Glossar 1}, description={Eine kurze Sammlung von Fachausdrücken}}
\newglossaryentry{abc}{name={Langer Glossary Namen 1}, description={Eine kurze Beschreibung}}
\newglossaryentry{def}{name={Langer Glossary Namen 2}, description={Eine lange Sammlung von Fachausdrücken, sogar eine sehr lange Sammlung von Fachausdrücken. Wow. Richtig lange Sammlung.}}
\newglossaryentry{g}{name={Sehr sehr langer Glossary Name, wow schau mal wie lang}, description={Eine lange Sammlung von Fachausdrücken. Eine lange Sammlung von Fachausdrücken. Eine lange Sammlung von Fachausdrücken}}
\newglossaryentry{test}{name={Test}, symbol={$\dagger$}, description={Eine lange Sammlung von Fachausdrücken. Eine lange Sammlung von Fachausdrücken. Eine lange Sammlung von Fachausdrücken. Eine lange Sammlung von Fachausdrücken.}}

\glsenablehyper
\makeglossaries
